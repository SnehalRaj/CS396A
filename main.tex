\documentclass[12pt]{report}
\usepackage[left=1in, right=1in, top=1in, bottom=1in]{geometry}
\usepackage[utf8]{inputenc}

\usepackage{amsmath}
\usepackage{amsthm}
\newtheorem{theorem}{Theorem}[section]
\newtheorem{lemma}{Lemma}[section]
\newtheorem{example}{Example}[section]
\newtheorem{definition}{Definition}[section]

\usepackage{tikz}
\usetikzlibrary{quantikz}

\title{Quantum Query Complexity}
\author{Akash Kumar Singh, Snehal Raj, Siddhant Kar}
\date{June 2020}

\begin{document}

\maketitle

\chapter{Introduction}
This report focuses on some recent work regarding a quantum query problem, namely FORRELATION. This problem has achieved the largest known gap between classical and quantum complexity yet among promise problems. There has also developed a new notion of block multilinear polynomials, which has helped in solving quantum query problems, including FORRELATION, in sub-linear classical time. We then explore this notion further by understanding block multilinear degree and its relation to standard degree.

\section{Quantum Query Model}
The quantum query model is used to answer some question about a boolean function, either partial (promise problem) or total (total problem). A partial function has some condition on the outputs and hence can be constructed using fewer inputs, whereas a total function has no such restriction. For example, Simon's problem is a promise problem and Grover's total search is total.

For our convenience, we assume these are functions from $\{-1,1\}^n$ to $\{-1,1\}$. Let us represent them in the standard basis as $x = (x_1, \ldots, x_N)$, where $N = 2^n$ and $\forall i$ $x_i \in \{-1,1\}$ corresponds to the $i$th output. We want to check a property $f$, which answers a question about $x$ with yes or no (decision problem). We have $$f : \{-1,1\}^N \longrightarrow \{0,1\}.$$

We are given access to an oracle or black box $O_x$ which is a quantum gate that, on input $|i\rangle$, does either of the following.

\begin{center}
\begin{quantikz}
\lstick[wires=2]{$\ket{i}$} & \gate[wires=3]{O_x} & \qw \rstick[wires=2]{$\ket{i}$} \\
&  & \qw \\
\lstick{$\ket{b}$} &  & \qw \rstick{$\ket{b \oplus x_i}$}
\end{quantikz}
\begin{quantikz}
\lstick[wires=2]{$\ket{i}$} & \gate[wires=3]{O_x} & \qw \rstick[wires=3]{$x_i\ket{i}\ket{b}$} \\
&  & \qw \\
\lstick{$\ket{b}$} &  & \qw
\end{quantikz}
\end{center}

Both the oracles shown above are equivalent. For example, to convert the first to the second, we set the control bit $|b\rangle$ to $|-\rangle$. We will only use the latter oracle that takes $|i\rangle$ and returns $x_i |i\rangle$. Our quantum query algorithm can call the oracle several times and also perform other linear operations in between as shown below.

\begin{center}
\begin{quantikz}
\lstick[wires=2]{$\ket{i}$}
& \gate[wires=4]{U_0} & \gate[wires=4]{O_x}
& \gate[wires=4]{U_1} & \gate[wires=4]{O_x}
& \ \ldots\ \qw
& \gate[wires=4]{U_{t-1}} & \gate[wires=4]{O_x}
& \gate[wires=4]{U_t} & \meter{} \\
&&&& & \ \ldots\ \qw &&&& \meter{} \\
\lstick[wires=2]{$\ket{w}$} &&&& & \ \ldots\ \qw &&&& \meter{} \\
&&&& & \ \ldots\ \qw &&&& \meter{}
\end{quantikz}
\end{center}

The gates in between the oracles are fixed and do not vary with $x$. Measuring the final qubits should result in an accepted state with high probability if $f(x)=1$ and with low probability if $f(x)=0$. The number of calls $t$ to the oracle $O_x$ is called the quantum query complexity of the algorithm.

\begin{definition}
A block multilinear polynomial on $\{-1,1\}^N$ is a polynomial of the form $$p(x) = p(x_{1,1}, x_{1,2}, \ldots) = \sum_{(i_1, \ldots, i_k)} a_{i_1 \ldots i_k} x_{1,i_1} \ldots x_{k,i_k}$$ where its $N$ variables can be partitioned into $k$ disjoint blocks $B_i, i \in [k]$ such that $x_{i,j} \in B_i\ \forall j$. 
\end{definition}

\begin{theorem}
The probability that a quantum query algorithm of complexity $t$ accepts a function $x$ is given by $p(x, x, \ldots)$, where $p$ is a block multilinear polynomial with $2t$ blocks of size $N$ each, and $p(\bar{x})$ is bounded in $[-1,1]$.
\end{theorem}
\begin{proof}
Consider the query model discussed above. Let $x_{j,i}$ denote the value of $x_i$ returned on the $j$th query to the oracle. We first prove that the amplitudes of the final state after $t$ queries are block-multilinear in $x_{1,1}, \ldots, x_{t,N}$. We prove this using induction on $t$.

When $t=0$, the amplitudes are simply given by constant polynomials. Let the result hold for $t-1$ queries. So now, the state of the qubits after the gate $U_{t-1}$ is given by
$$\sum_{i,w} a_{i,w}(x_{1,1}, \ldots, x_{t-1,N}) |i\rangle |w\rangle.$$
After applying $O_x$, the state is
$$\sum_{i,w} x_{t,i} a_{i,w}(x_{1,1}, \ldots, x_{t-1,N}) |i\rangle |w\rangle.$$

It is easy to see that the amplitudes are still block-multilinear, now with $t$ blocks, even after the linear transformation $U_t$. The squares of the amplitudes are also block-multilinear, but with $2t$ blocks. We simply introduce a duplicate variable $x_{t+j,i}$ for each $x_{j,i}$ and then square amplitudes as shown below. The final polynomial $p(\bar{x})$ is given by
$$\sum_{i,w \in Acc} |a_{i,w}(x_{1,1}, \ldots, x_{t,N})|^2 = \sum_{i,w \in Acc} a^*_{i,w}(x_{1,1}, \ldots, x_{t,N}) a_{i,w}(x_{t+1,1}, \ldots, x_{2t,N})$$
which is clearly block multilinear. The probability that the algorithm accepts is simply $p(x, x, \ldots)$, which means that the norm of the vector of accepted state amplitudes is at most 1. Thus, $p(\bar{x})$ being an inner product lies in $[-1,1]$.
\end{proof}

\section{Forrelation}
Forrelation is a promise problem which measures the correlation between a function $f$ and the fourier transform of a second function $g$.

\end{document}
