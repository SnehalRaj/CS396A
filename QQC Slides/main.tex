\documentclass[10pt]{beamer}
\usetheme[progressbar=frametitle]{metropolis}
\usepackage{appendixnumberbeamer}
\usepackage{ragged2e}
\usepackage{bm}
\justifying
\usepackage[numbers,sort&compress]{natbib}
\bibliographystyle{plainnat}

\usepackage{booktabs}
\usepackage[scale=2]{ccicons}
\usepackage{amsmath}
\newcommand{\Exp}{\mathrm{E}}
\newcommand{\Var}{\mathrm{Var}}
\newcommand{\Cov}{\mathrm{Cov}}
\newcommand{\bmdeg}{\mathrm{bmdeg}}
\newcommand{\sgn}{\mathrm{sgn}}

\usepackage{amssymb}
\usepackage{commath}
\usepackage{mathtools}
\DeclarePairedDelimiter\ceil{\lceil}{\rceil}
\DeclarePairedDelimiter\floor{\lfloor}{\rfloor}

\usepackage{xspace}
\newcommand{\themename}{\textbf{\textsc{metropolis}}\xspace}
\usepackage{graphicx}
\graphicspath{ {./images/} }
\title{Block Multilinear Degree}
\subtitle{Paper Review}
\date{}
\author{Akash Kumar Singh, Siddhant Kar, Snehal Raj}
\institute{IIT Kanpur}

\begin{document}
\maketitle

\begin{frame}{Table Of Contents}
  \setbeamertemplate{section in toc}[sections numbered]
  \tableofcontents[hideallsubsections]
\end{frame}

\section{Introduction}

\begin{frame}[fragile]{Quantum Query Model}
\begin{itemize}
    \item The quantum query model is the most widely used model to study quantum algorithms.
    \item Given a boolean function $x\colon \{-1,1\}^n\rightarrow \{-1,1\}$, we are required to calculate a property $f(x)$, where $f\colon \{-1,1\}^N\rightarrow \{0,1\}$ and $N = 2^n$.
    \item An algorithm based on this model is allowed to make several queries to $x$ in order to calculate $f(x)$.
    \item The minimum number of queries that any such algorithm must make in order to determine $f$ is called the quantum query complexity $Q_f$ of $f$.
\end{itemize}
\end{frame} 

\begin{frame}{Quantum Query Model}
\begin{itemize}
    \item The polynomial method puts a lower bound on $Q_f$.
    \item Using the method, we can construct a degree $2Q_f$ polynomial that approximates $f$ up to some $\epsilon$.
    \item The minimum degree that any such polynomial can attain is called the $\epsilon$-approximate degree of $f$, denoted $\widetilde{\deg}_\epsilon(f)$.
    \item Thus, $2Q_f \geq \widetilde{\deg}_\epsilon(f)$.
\end{itemize}
\end{frame} 

\begin{frame}{Block-multilinear polynomials}
\begin{itemize}
    \item Aaronson et al. \cite{paper1} introduced a new notion and approximated $f$ up to $\pm \epsilon$ using a "block-multilinear" polynomial of degree $2Q_f$.
    \item A block multilinear polynomial on $\{-1,1\}^n$ is of the form $$p(x) = p(x_{1,1}, x_{1,2}, \ldots) = \sum_{(i_1, \ldots, i_k)} a_{i_1 \ldots i_k} x_{1,i_1} \ldots x_{k,i_k}$$ where its $n$ variables can be partitioned into $k$ disjoint blocks $B_i, i \in [k]$, such that $x_{i,j} \in B_i\ \forall j$.
    \item The minimum degree attainable by a block-multilinear polynomial that approximates $f$ up to $\pm \epsilon$ is called the $\epsilon$-approximate block-multilinear degree, denoted $\widetilde{\bmdeg}_\epsilon(f)$.
    \item We thus have $2Q_f \geq \widetilde{\bmdeg}_\epsilon(f)$.
\end{itemize}
\end{frame}

\begin{frame}{Forrelation}
\begin{itemize}
    \item This notion of block-multilinear polynomials is used in \cite{paper1} to solve a problem called Forrelation.
    \item It is a measure of the correlation between a function $f$ and the fourier transform of a second function $g$.
    \item Given oracle access to two boolean functions $f,g\colon \{0,1\}^{n} \rightarrow  \{-1,1\}$, let
    \begin{equation*}
    \Phi_{f, g} := \frac{1}{2^{3n/2}} \sum_{x, y \in \{0,1\}^{n}} f(x)(-1)^{x \cdot y} g(y).
    \end{equation*}
    \item We have to decide whether $\Phi_{f,g} \geq 0.6$ or $\left|\Phi_{\mathrm{f}, \mathrm{g}}\right| \leq 0.01$, promised that one of them is the case.
\end{itemize} 
\end{frame}

\section{Block-multilinear Degree v/s Degree}
\begin{frame}{Bmdeg vs Deg}
\begin{itemize}
    \item We want compare the exact block-multilinear degree of a Boolean function to its degree.
    \item Block-multilinear degree is at least equal to the degree.
    \item We want to find $f$ for which the inequality is strict, that is, $\widetilde{\bmdeg}_\epsilon(f) > \widetilde{\deg}_\epsilon(f)$.
    \item Approach 1: Construct a block-multilinear polynomial by symmetrically splitting the coefficients of a given polynomial.
    \item Approach 2: Find a dual witness for a suitable linear program.
\end{itemize}
\end{frame}

\begin{frame}{Symmetrization}
\begin{itemize}
    \item Given an exact polynomial representation of $f$ in the Fourier basis
    \begin{equation*}
    f(x) = \sum_{S\subseteq [n]} \hat{f}(S) \chi_S(x)
    \end{equation*}
    where $\chi_S(x) = \prod_{i\in S}x_i$.
    \item Construct a symmetric block-multilinear polynomial by symmetrically splitting $\hat{f}(S) \chi_S(x)$.
    \item Check whether this construction is bounded in $[-1,1]$ on all possible inputs using an Integer Linear Program.
    \item Found a counter-example where the polynomial was not bounded.
\end{itemize}
\end{frame}

\begin{frame}{Dual Witness}
\begin{itemize}
    \item Consider the linear program that finds the best possible approximation of a function $f\colon \{-1,1\}^N \rightarrow \{0,1\}$ using block-multilinear polynomials $g$ of degree $d = \deg(f)$.
    \item Weak duality implies that if the value of the dual is strictly greater than $\epsilon_0$, then the value of the primal is greater than $\epsilon_0$ as well.
    \item This implies that $\bmdeg_{\epsilon_0}(f) > d$.
\end{itemize}
\begin{theorem}
Let $f\colon \{-1,1\}^N \rightarrow \{0,1\}$ have degree $d$. Then $\bmdeg_{0}(f) > d$ if and only if there exist $\phi\colon \{-1,1\}^N \rightarrow \mathbb{R}$ and $\psi_1, \psi_2\colon \{-1,1\}^{(N+1)d} \rightarrow \mathbb{R^+}$ such that \begin{enumerate}
    \item $\sum_x \phi(x)f(x) \geq \sum_{\bar{x}} (\psi_1(\bar{x}) + \psi_2(\bar{x})).$
    \item ${\hat{\psi}}_1(m) - {\hat{\psi}}_2(m) = \frac{N}{2^{(N+1)d}} \hat{\phi}(S_m) ~\forall m\in \{0, \ldots, N\}^d.$
\end{enumerate} 
\end{theorem}
\end{frame}

\section{Classical-Quantum Gap}

\begin{frame}{Classical-Quantum Gap}
\begin{itemize}
    \item Aaronson et al. \cite{paper1} showed that Forrelation for $k$ functions can be solved in $\lceil k/2 \rceil$ quantum queries.
    \item They also showed that it requires at least $\Omega(\frac{\sqrt{N}}{\log N})$ classical queries.
    \item Forrelation thus has a gap of $\Omega(\frac{\sqrt{N}}{\log N})$ between its quantum and classical query complexities.
    \item It is the largest gap known yet among promise problems.
\end{itemize}
\end{frame}

\begin{frame}{Real Forrelation}
We first convert the Forrelation problem into a Real Forrelation problem. In Real Forrelation, we are given oracle access to two real functions $f,g\colon \{0,1\}^{n}\rightarrow \mathbb{R}$ and are promised that either
\begin{enumerate}
    \item every $f(x)$ and $g(y)$ value is an independent $\mathcal{N}(0,1)$ Gaussian, or else
    \item every $f(x)$ value is an independent $\mathcal{N}(0,1)$ Gaussian and every $g(y)$ value equals $\hat{f}(y)$ (i.e. the Fourier transform of $f$ evaluated at $y$).
\end{enumerate}
\end{frame}

\begin{frame}{Gaussian Distinguishing}
In the Gaussian Distinguishing problem, we are given oracle access to a collection of $\mathcal{N}(0,1)$ real Gaussian variables $x_{1},x_{2}..., x_{m}$ and are asked to decide whether 
\begin{enumerate}
    \item the variables are all independent, or
    \item the variables lie in a known low dimensional subspace $S \leq R^{m}$ such that there is a covariance of at most $\epsilon$ between each pair of variables, i.e., $ |~Cov(x_{i},x_{j})~| \leq \epsilon ~\forall~ i,j$.
\end{enumerate}
\end{frame}

\begin{frame}{Randomized Lower bound}
\begin{itemize}
    \item If there exists a $T$-query algorithm that solves Forrelation with bounded error, then there also exists an $O(T)$-query algorithm that solves Real Forrelation with bounded error.
    \item Gaussian distinguishing requires $\Omega\left(\frac{1 / \varepsilon}{\log (M / \varepsilon)}\right)$ classical randomized queries.
    \item Using the above stated results, any classical randomized algorithm for must make $\Omega(\frac{\sqrt{N}}{\log N})$ queries.
\end{itemize}
\end{frame}

\begin{frame}{Optimized Randomized Algorithm}
\begin{itemize}
    \item Forrelation requires at least $\Omega(\sqrt{N}/\log n)$ queries classically but just one quantum query.
    \item This 1-query quantum algorithm can be converted to a $\sqrt{N}$-query randomized algorithm.
    \item  We do this by using an estimator to estimate the block-multilinear polynomial.
\end{itemize}
\end{frame}

\begin{frame}[allowframebreaks]
    \frametitle{}
    \bibliography{refs}
\end{frame}

\end{document}
